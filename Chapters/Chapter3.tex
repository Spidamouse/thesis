% Chapter Template

\chapter{Methods and Materials} % Main chapter title

\includegraphics[width=\linewidth,trim={0 13cm 0 6cm},clip]{Paradise_Lost_47}

\label{Chapter 3} % Change X to a consecutive number; for referencing this chapter elsewhere, use \ref{ChapterX}

\section{Overview}

This chapter describes how work on the project was conducted. It does this by first describing the methodology used, then by following with the hardware and software used. Finally, an overview of the thoughts about this chapter will be given.

%----------------------------------------------------------------------------------------
%	SECTION 1
%----------------------------------------------------------------------------------------

\section{Methodology}

\label{Ch3 Sec1}

This project included several tasks, in which, debugging skills and a deep programming knowledge base was necessary. While debugging there were a few techniques that were used to enabled fast and more accurate pinpointing of the offending pieces of hardware. Chief among these debugging techniques was to visually expose relevant signals. By exposing a signal in real time it could be quickly and easily verified as either working or not working. Should this fail, another useful debugging technique was to simulate the modules with report statements. While this was not always viable, it did allow for much more real time data to be exposed; it also allowed exact inputs to the hardware to be monitored. The final debugging technique used was to reduce the complexity of the hardware. By removing more complex elements of the hardware it is possible to isolate the issue and then fix it.

\section{Materials}

\label{Ch3 Sec2}

%-----------------------------------
%	SUBSECTION 1
%-----------------------------------
\subsection{Hardware Used}

\label{Ch3 Sec2 Sub1}

During the development of this project multiple hardware devices were used, all of this hardware enabled either the creation, interaction or programming of the MEGA65. In addition to this some of the work on the project required specific hardware in order for full functionality of the MEGA65 to be present. This hardware is listed as follows:\\

\begin{itemize}
\item{Digilent Nexys 4 DDR Artix-7 FPGA Trainer Board: This field programmable gate array (FPGA) development board, as seen in figure \ref{fig:nexys4ddr}, was one of the vital components while conducting this project. It was used as the remappable hardware, on which, the MEGA65 prototypes could be developed and tested.}
\item{Acer Aspire V Nitro: This laptop, as seen in figure \ref{fig:aspirevnitro}, was the most important piece of hardware used in this project. This piece of hardware was used to interface with the FPGA, as well as make changes to the MEGA65 hardware.}
\item{Dell UltraSharp 2408WFP 24-inch LCD Monitor: This monitor, while not critical, was a required piece of hardware. This was used to host the VGA output from the MEGA65.}
\item{Dell KB1421 Keyboard: Much like the monitor, this piece of hardware was not critical but without it, or a similar product, progress on the project would be exceedingly difficult.}
\item{SanDisk Ultra 16Gb MicroSDHC Micro SD card: This SD card, like the keyboard and monitor, was not critical, but it was necessary for full operation of the MEGA65 however.}
\end{itemize}

\begin{figure}
  \centering
  \includegraphics[width=0.75\linewidth]{nexys4ddr}
  \caption{The Nexys 4 DDR Artix-7 FPGA Trainer Board}
  \label{fig:nexys4ddr}
\end{figure}

\begin{figure}
  \centering
  \includegraphics[width=0.75\linewidth]{aspirevnitro}
  \caption{The Acer Aspire V Nitro Laptop}
  \label{fig:aspirevnitro}
\end{figure}

%-----------------------------------
%	SUBSECTION 1
%-----------------------------------
\subsection{Software Used}

\label{Ch3 Sec2 Sub2}

During development of the MEGA65, several software packages were used in order to interact with the hardware listed above. These packages include:\\

\begin{itemize}
\item{Ubuntu 18.04: This operating system was required for work on the project. It was through it that commands required to program and interact with the FPGA were able to function.}
\item{Vivado HLx Webpack Edition: The development tools that this webpack provided were critical for the function of the MEGA65. These tools were used to program and make changes to the FPGA.}
\item{Git: This version control tool, while not necessary, was extremely useful. This allowed changes to be tracked and shared between developers on the project. The hosted repositories are found on the GitHub website}
\item{GHDL: This VHDL simulation software was very useful when debugging during the project. It allowed for more easily accessible signals during run time.}
\end{itemize}

\section{Final Thoughts}

During this project several methods of development were used, most useful among these methods was to expose the signals physically on the FPGA. This allowed for clear confirmation of the states of the various finite state machines on the board. Secondary to this the GHDL simulation of the hardware also proved useful as it allowed greater exposure of the inner signals of the FPGA.\\
During this project several hardware and software packages were used. The most important hardware packages consisted of the FPGA, keyboard, monitor and computer. The key software packages included the Ubuntu Linux operating system, the hardware development software Vivado and the source control package Git.
