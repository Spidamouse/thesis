% Chapter Template

\chapter{Introduction} % Main chapter title

\label{Chapter 1} % Change X to a consecutive number; for referencing this chapter elsewhere, use \ref{ChapterX}

In this introductory chapter the context of this honours thesis will be provided, and the aims of the thesis will be stated.
In the following section, section \ref{Ch1 Sec1}, the need for this thesis is justified.
Following that, in section \ref{Ch1 Sec2}, the issue will be outlined in the form of a problem statement.
This will then be translated into objectives for this specific thesis and will then be followed by the research questions.
After the research questions, section \ref{Ch1 Sec3} will further describe how the rest of this document will be laid out.   

%----------------------------------------------------------------------------------------
%	SECTION 1
%----------------------------------------------------------------------------------------

\section{Motivation}

\label{Ch1 Sec1}

In the current digital age, many people are becoming increasingly anxious about the security of communication devices.
This concern is not without reason. One the one hand, there is a seemingly unending stream of hardware and software vulnerabilities \cite{WannaCry}\cite{SpecMel}.
Then on the other hand, the concept of privacy in the digital world is continually being erroded, including by governments seeking to increase their surveillance and related powers in the name of national security \cite{Reference37}.

However, even lawfully legislated powers are not with out risk.  For example, in 2013, it was discovered that the Australian Security Intelligence Organisation (ASIO) had been monitoring civilian phones due to an "error" when taking down the phone number, causing the digits to be misordered, compromising the privacy of the owners of the incorrectly entered phone numbers.
Due to "technical difficulties" the data that was harvested was retained for 10 months before finally being destroyed \cite{Reference36}.
Events like these remind us that even well-intentioned government agencies are not infallible, and that it is not unjustified to seek to improve the intrinsic privacy of mobile telephones.

Additionally, the concept of what is private data is also under attack.
Earlier this year the very definition of what is private information was reduced by the federal court.
This results in information, such as cell tower location, inbound call data and IP addresses, no longer being covered by the privacy act, thus giving the government greater ability to monitor it's citizens without probable cause \cite{Reference37}.\\
With both warranted and unwarranted data collection being done through mobile phones, there is a clear and present danger of a slippery slope towards an Orwellian state, as governments accrete power little-by-little, until the necessary checks and balances on the power of political parties and governments are eroded, and the temptation to retain power at all costs leads to the breakdown of freedom and democracy.  The increasing monitoring and control of the population of China through digital gives a proof-by-example of how this can eventuate.

Therefore, seeing the combination of current events and recalling the history of countries like East Germany who have lived under a pervasive police state, there is a clear need to ensure that the law abiding populace of liberal democracies are able to communicate freely with one another in the digital age, precisely so that democracies and the freedoms that liberal democracies exist to provide can be protected.
This thesis explores this issue by considering how a computer based communications device can be made secure in the 21st century.

%----------------------------------------------------------------------------------------
%	SECTION 2
%----------------------------------------------------------------------------------------

\section{Scope}

\label{Ch1 Sec2}

%-----------------------------------
%	SUBSECTION 1
%-----------------------------------

\subsection{Problem Statement}

\label{Ch1 Sec2 Sub1}

Everyone has personal data that they wish to keep from prying eyes.
Whether the secrecy of this information is important at a national or economic level, such as banking details, protecting ones location from abusive ex-partners, or simple secrets, such as planning a surprise party, it doesn't matter.
The importance of the integrity of private correspondence is perhaps best expressed in Article 10 of the German Basic Law \cite{Germany}, the German equivalent of a constitution, that states simply that "the privacy of correspondence, posts and telecommunications shall be inviolable".
That is, information should be able to be communicated electronically, without fear of disclosure, manipulation or other misadventure. By extension, it must be possible to store such correspondence electronically.
However, in the 21st century the inviolability of telecommunications has become increasingly impossible: The hardware and software vulnerabilities of software mean that systems are almost always vulnerable to lawful and unlawful interception, because of lack of fundamental security.

Worse, the complexity of modern computer systems means that there is no way of verifying that data has not been compromised, or that data is truly safe from compromise, because it is impossible to verify the correct operation of such systems.
The recent practice of adding security features is not a solution to this problem, because each new feature increases the complexity, and thus makes verification all the more impossible. The vulnerabilities identified in the Intel Management Engine is a good example of the inherent irony of the current situation:
Measures intended to keep computers safe have the opposite effect in practice, precisely because the increase the attack surface, rather than decrease it \cite{IME}.

The problem is therefore that modern computing and communications devices have become too complex to be verified, and therefore too complex to trust.  Therefore this thesis examines how much simpler systems can be designed, simple enough for a single determined user to verify them, so that we can reach the point where the inviolability of telecommunications becomes a reality.


%-----------------------------------
%	SUBSECTION 2
%-----------------------------------

\subsection{Objectives}

\label{Ch1 Sec2 Sub2}

Within the overall objective of creating a simple, secure smart-phone like device, it is hope that the following will be accomplished in this project:

\begin{itemize}
\item Secure inter-process communication is established.
\item The sub-systems of the phone are evaluated for their readiness.
\item The unready sub-systems are completed.
\item The architecture as a whole is evaluated at a high-level for completeness and security, ready for being handed over to the next phase of development.
\end{itemize} 

%-----------------------------------
%	SUBSECTION 3
%-----------------------------------

\subsection{Research Questions}

\label{Ch1 Sec2 Sub3}

What are the missing or non-functional sub-systems that the MEGA65 requires to be implemented?\\*
Which will be answered in the form of a survey of the current sub-systems of the MEGA65, and a survey of the sub-systems required to create a functional prototype.\\
\\
How can these sub-systems be implemented?\\
Which will be answered by the creation of plans for the implementing of the missing or incomplete sub-systems.\\
\\
How can the simplicity, understandability and hence, the security of these sub-systems be maximised?\\
Which will be answered by considering each sub-system, qualitatively appraising its simplicity, understandability and security, and where appropriate, making well researched recommendations for refining those components to improve one or more of these axes, and time permitting, acting on those recommendations.\\
\\
How can the complete MEGA65 architecture be physically prototyped on the bench?\\
Which will be answered by examination of the current partial bench prototype and comparing it with the sub-systems identified through the other research questions, and designing and realising a complete bench prototype.
This will occur through coordination with Mr. Lachlan McDonald, who is undertaking the designing of the PCB for the MEGA65 smart-phone device.\\
\\
How can the secure compartmentalisation's architecture planned for the MEGA65 be realised?\\
Which will be answered by considering this architecture and the current state of the MEGA65 system to derive and execute a method for implementing a secure compartmentalisation architecture.\\
\\
Overall the success of the project will be measured against the creation of a functioning bench prototype device that, through the architecture, implements the secure and understandable compartmentalisation of hardware to the point of demonstrability.

%----------------------------------------------------------------------------------------
%	SECTION 3
%----------------------------------------------------------------------------------------

\section{Thesis Layout}

\label{Ch1 Sec3}

Following this chapter, "Chapter \ref{Chapter 1} : Introduction", is the chapter, "Chapter \ref{Chapter 2} : Literature Review".
In which relevant background information regarding complexity and cyber security, mobile devices, isolative security and the MEGA65 project will be given.
After this, "Chapter \ref{Chapter 3} : Methods and Materials" will go on to describe how this project was undertaken and what tools were used.
This will be immediately followed by "Chapter \ref{Chapter 4} : Project Set-up".
In this chapter, details about the issues faced immediately after joining the project will be made clear.
Following this is "Chapter \ref{Chapter 5} : Matrix Mode Corrections", in this chapter the issues encountered with the matrix mode and their fixes will be made clear.
"Chapter \ref{Chapter 6} : Secure Compartmentalisation" will follow, in which an overview of how the secure containers in the phone were implemented will be given.
Then, "Chapter \ref{Chapter 7} : Bug Fixing" will outline issues with and describe the corrections made to the various subsystems of the phone.
This will be followed by "Chapter \ref{Chapter 8} : Results and Discussion" where XXX will be talked about.
Finally, "Chapter \ref{Chapter 9} : Conclusion" will discuss the various details about XXX as well as the future for the project.

